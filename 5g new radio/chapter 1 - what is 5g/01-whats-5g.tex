\documentclass{beamer}
\usepackage[utf8]{inputenc}
\usepackage{amsmath, amssymb, amsthm, mathtools, gensymb}

\setbeamertemplate{footline}[frame number]
\beamertemplatenavigationsymbolsempty
\usecolortheme{seagull}

\newcommand{\tran}{\mathsf{T}}
\newcommand{\mc}[1]{\ensuremath{\mathcal{#1}}}
\newcommand{\mbb}[1]{\ensuremath{\mathbb{#1}}}	

\title[Chapter 7]{Notes on Dahlmann, Parkvall, Skold's ``5G NR''\\Chapter 1 --  ``What's 5G''}
\author{Lucas N. Ribeiro}
\date{}

\begin{document}
	
	\frame{\titlepage}
	
	\begin{frame}[allowframebreaks]{Introduction}
	Different generations of mobile communications
	\begin{itemize}
		\item 1980s: foundations of mobile telephony (1G);
		\item 1990s: mobile telephony for everyone (2G);
		\item 2000s: foundations of mobile broadband (3G);
		\item 2010s: enhanced mobile broadband (4G - LTE);
	\end{itemize}
	
	The next generation  -- 5G/NR -- Use cases
	\begin{itemize}
		\item enhanced mobile broadband (eMBB): enhancing mobile broadband, increasing rates, capacity, etc of today's networks;
		\item massive machine-type comm (mMTC): massive numbers of devices (sensors) connected. Requirements: low energy consumption and cost, high data rate not very important. Comm. foundation for the internet of things;
		\item ultra reliable and low-latency communication (URLLC): very low latency and extreme reliability for factory automation, traffic safety, etc
	\end{itemize}
	
	New Radio -- The new radio access technology -- NR reuses many features of LTE , however it is not restricted by a need to maintain backwards compatibility.
	
	5GCN -- The new 5G core network to whom NR will connect.
	
\end{frame}


\end{document}

