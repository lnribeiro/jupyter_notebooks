\documentclass{beamer}
\usepackage[utf8]{inputenc}
\usepackage{amsmath, amssymb, amsthm, mathtools, gensymb}

\setbeamertemplate{footline}[frame number]
\beamertemplatenavigationsymbolsempty
\usecolortheme{seagull}

\newcommand{\tran}{\mathsf{T}}
\newcommand{\mc}[1]{\ensuremath{\mathcal{#1}}}
\newcommand{\mbb}[1]{\ensuremath{\mathbb{#1}}}	

\title[Chapter 7]{Notes on Dahlmann, Parkvall, Skold's ``5G NR''\\Chapter 2 --  ``5G standardization''}
\author{Lucas N. Ribeiro}
\date{}

\begin{document}
	
	\frame{\titlepage}
	
	\begin{frame}[allowframebreaks]
		\textbf{ Type of organizations involved in creating technical specifications and standards, as well as regulations:}
		\begin{itemize}
			\item Standards Developing Organizations (SDOs): 3GPP, ETSI, ATIS, etc;
			\item Regulatory Bodys and Administrations: control spectrum use, set licenses, regulate placing on the market, award certifications, \emph{setting requirements}. Examples: ANATEL, ECC (Nationals), ITU (Global);
			\item Industry forums: promoting and lobbying for specific technologies. GSMA Association, Next Generation Mobile Networks, etc
		\end{itemize}
	
		\newpage
		\textbf{The role of ITU-R:}
		\begin{itemize}
			\item It's the radio communications sector of ITU;
			\item Aims at ensuring efficient and interference-free use of RF spectrum by all wireless devices;
			\item Composed of subgroups and working parties that produce reports and recommendations;
			\item Within ITU-R is the Working Party 5D (WP5D), which is responsible for the overall system aspects of International Mobile Telecommunications (IMT) systems
			\item Provides \emph{Radio Interface Specifications} (RSPCs);
			\item Each recommendation contains \emph{Radio Interface Technologies} (RITs);
			\item It does not create specifications, but maintains recommendations and reports for IMT;
			\item The actual specifications are maintained by an SDO and the RSPCs provide references to the specifications
			\item Examples of RSPC:
			\begin{itemize}
				\item IMT-2000: (3G, WCDMA, etc);
				\item IMT-Advanced:  4G/LTE
				\item IMT-2020 (planned): will provide radio interface technologies for 5G.
			\end{itemize}
		\end{itemize}
	
		\textbf{IMT Process in ITU-R WP5D}
		\begin{enumerate}
			\item Looks for future roles and trends;
			\item Set future usage scenarios (eMBB, URLLC, mMTC)
			\item At World Radio Conference (WRC) 2015 new bands for IMT were discussed;
			\item Publication of technical requirements, evaluation guideline, submission template (for candidate technology)
		\end{enumerate}
	
		\textbf{Capabilities of IMT-2020}
		\begin{itemize}
			\item 13 capabilities, in which 8 are \emph{key capabilities}
			\item Key capabilities
			\begin{enumerate}
				\item peak data rate (bandwidth times peak spectral efficiency)
				\item user experienced data rate (95th percentile of the user data rate dist.) -- around 1 Gbps at 5G
				\item spectrum efficiency
				\item area traffic capacity
				\item network energy efficiency
				\item latency
				\item mobility
				\item connection density
			\end{enumerate}
			\item The 5 first capabilities are very important for the Enhanced Mobile Broadband usage scenario
			\item Additional capabilities
			\begin{enumerate}
				\item spectrum and bandwidth flexibility
				\item reliability -- high level of availability
				\item resilience -- ability to work correctly after man-made or natural disturbance, such as loss of power
				\item security and privacy
				\item operational lifetime (energy capacity)
			\end{enumerate}
		\end{itemize}
	
		\newpage
		\textbf{IMT-2020 performance requirements and evaluation}
		\begin{itemize}
			\item test environments:
			\begin{itemize}
				\item Indoor hotspot - eMBB - indoor isolated environment (offices, etc) with very high user density
				\item Dense urban - eMBB - urban scenario with high user density
				\item Rural - eMBB - wide coverage supporting pedestrian and high-speed vehicles
				\item Urban macro - mMTC - coverage of high number of connected machine type devices
				\item Urban macro - URLLC - urban macro scenario for ultra reliable and low latency comms.
			\end{itemize}
			\item Evaluation methods
			\begin{itemize}
				\item Simulation -- system and/or link level simulations. 
				\item Analysis
				\item Inspection
			\end{itemize}
		\end{itemize}
	
		\textbf{3GPP standardization}
		\begin{itemize}
			\item Organization responsible for the specification of technologies that aim to achieve the IMT requirements
			\item 2G GSM, 3G WCDMA/HSPA, 4G LTE, 5G NR
			\item To understand how 3GPP works, it's important to understand the process of standard writing
			\begin{itemize}
				\item Requirements -- what to achieve with the specification?
				\item Architecture -- main building blocks defined
				\item Detailed specifications
				\item Testing and verification
			\end{itemize}
			\item 3GPP (Project coordination group)
			\begin{itemize}
				\item Technical Specification Groups - TSG (Radio Access Network, etc)
				\begin{itemize}
					\item Working groups (WG) -- (RAN WG1 - Radio layer 1)
				\end{itemize}
			\end{itemize}
			\item 3GPP technical specifications (TS) are numbered as TS XX.YYY where XX denotes the series and YYY its number within the series, ex:
			\begin{itemize}
				\item 36-series: LTE
				\item 38-series: NR
			\end{itemize}
		\end{itemize}
	\end{frame}
\end{document}

