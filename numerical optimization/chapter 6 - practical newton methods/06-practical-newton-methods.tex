\documentclass{beamer}
\usepackage[utf8]{inputenc}
\usepackage{amsmath, amssymb, amsthm, mathtools, gensymb}

\setbeamertemplate{footline}[frame number]
\beamertemplatenavigationsymbolsempty
\usecolortheme{seagull}

\newcommand{\tran}{\mathsf{T}}
\newcommand{\mc}[1]{\ensuremath{\mathcal{#1}}}
\newcommand{\mbb}[1]{\ensuremath{\mathbb{#1}}}	

\title[Chapter 6]{Notes on Nocedal and Wright's ``Numerical Optimization''\\Chapter 6 --  ``Practical Newton Methods''}
\author{Lucas N. Ribeiro}
\date{}
 
\begin{document}
 
\frame{\titlepage}
 
\begin{frame}[allowframebreaks]{Introduction}
	\begin{itemize}
		\item Line search methods:
		\[
			x_{k+1} = x_k + \alpha_k p_k
		\]
		\item $p_k$ is the search direction (must point to descent direction -- leading to global convergence!)
		\item Gradient descent: $p_k = -\nabla f_k$.
		\item More general: $p_k = -B_k^{-1} \nabla f_k$, where $B_k$ is a symmetric and non-singular matrix.
		\item Newton method: $B_k = \nabla^2 f_k$ (Hessian).
		\item Problems with standard Newton method: Hessian may not be positive definite all the time, and/or close to singular, possibly leading to ascent directions!
		\item The Newton step is obtained by solving the symmetric $n\times n$ linear system:
		\[
			\nabla^2 f(x_k) p_k^N = -\nabla f(x_k)
		\]
		\item Approaches to ensure good Newton step:
		\begin{enumerate}
			\item Solve the Newton problem by means of a conjugate gradient (Newton-CG)
			\item Modifying the Hessian matrix $\nabla^2 f(x_k)$ so it becomes sufficiently positive definite (modified Newton method)
		\end{enumerate}
	
		\item Computational complexity concerns:
		\begin{enumerate}
			\item Newton-CG method -- terminate CG iteration before an exact solution is found (inexact Newton approach)
			\item Computation of the Hessian matrix is quite complex
		\end{enumerate}
	\end{itemize}
\end{frame}

\begin{frame}[allowframebreaks]{Inexact Newton steps}
	\begin{itemize}
		\item Standard Newton method has to calculate its step exactly, which can be expensive
		\item Classical linear system methods (Gaussian elimination etc) may be too complex when the number of variables is large $\rightarrow$ iterative solutions are appealing here
		\item An inexact solution may work
		\item Here we consider the CG residual $r_k = \nabla^2 f(x_k) p_k + \nabla f(x_k) $
		\item (Note that the standard CG residual is $r_k = Ax_k -b$!)
		\item A typical condition checked to terminate the iterative solver is 
		\[
			\|r_k\| \leq \eta_k \| \nabla f(x_k) \|
		\]
		where $\eta_k$ is the forcing sequence. This factor is important to determine the rate of convergence of inexact Newton methods.
	\end{itemize}
	\vspace{0.5cm}
	Rate of convergence results
	\begin{itemize}
		\item Local convergence is obtained simply by ensuring that $\eta_k$ is bounded away from $1$.
		\item Rate of convergence is linear if $\eta_k \rightarrow 0$ and superlinear if $\eta_k = O(\| \nabla f(x_k) \|)$
	\end{itemize}
\end{frame}

\begin{frame}[allowframebreaks]{Line search Newton methods}
	\begin{itemize}
		\item Each iteration has the form $x_{k+1} = x_k + \alpha_k p_k$
		\item $alpha_k$ is a step length that satisfied Wolfe-Goldstein conditions and can be obtained by backtracking (although $\alpha_k=1$ is a good initial guess)
		\item $p_k$ is either the exact Newton step or its approximation
	\end{itemize}

	Line search Newton CG
	\begin{itemize}
		\item Applies CG to calculate the Newton step.
		\item Recall that CG is designed to positive definite systems, but Hessian may admit negative eigenvalues, thus a test is introduced in the Newton CG to check for negative curvatures
		\item Preconditioning can be applied to correct the system's curvature.
	\end{itemize}

	\begin{block}{Line Search Newton CG}
		\begin{itemize}
			\item Given initial point $x_0$
			\item For ...
			\begin{enumerate}
				\item Compute search direction applying CG method to $\nabla^2 f(x_k) p = -\nabla f(x_k)$. Terminate if $\|r_k\| \leq \min (0.5, \sqrt{\nabla f(x_k)}) \|\nabla f(x_k)\|$ or if negative curvature is encountered.
				\item Find step size $\alpha_k$
				\item Set $x_{k+1} = x_k + \alpha_k p_k$ 
			\end{enumerate}
		\end{itemize}
	\end{block}
\end{frame}

\begin{frame}[allowframebreaks]{Modified Newton's method}
\begin{itemize}
	\item The search direction can be obtained by linear algebra factorization techniques.
	\item If the Hessian is not definite positive, we can modify this by adding a matrix $E_k$ so that it becomes nice.
	\item Global convergence can be attained if $E_k$ satisfies the bounded modified factorization property
	\item In this case $B_k = \nabla^2 f(x_k) + E_k$ has bounded condition number
	\item We discuss some Hessian modifications:
	\begin{itemize}
		\item Eigenvalue modification: changes negative eigenvalues to non-negative
		\item Adding multiple of identity
		\item Modified Cholesky factorization, Gershgorin modification: adds elements to the Cholesky diagonal
	\end{itemize}
\end{itemize}

\begin{block}{Line Search Newton CG}
	\begin{itemize}
		\item Given initial point $x_0$
		\item For ...
		\begin{enumerate}
			\item Factorize $B_k = \nabla^2 f(x_k) + E_k$
			\item Solve $B_k p_k = -\nabla f(x_k)$
			\item Find step size $\alpha_k$
			\item Set $x_{k+1} = x_k + \alpha_k p_k$ 
		\end{enumerate}
	\end{itemize}
\end{block}
\end{frame}

\begin{frame}{Trust-region Newton method}
	\begin{itemize}
		\item Trust-region methods do not require the quadratic model to be positive definite, then we can use $B_k = \nabla^2 f(x_k)$ and get $p_k$ by solving
		\[
		\min_{p} m_k(p)= f_k + \nabla f_k^\tran p + 0.5 p^\tran B_k p
		\]
		subject to $\| p \| \leq \Delta_k$.
		\item It can be solved by many techniques, including:
		\begin{itemize}
			\item Newton dogleg method
			\item 2D subspace minimization
			\item Accurate solution using iteration
			\item CG method (CG-Steihaug!)
			\item Lanczos method
		\end{itemize}
	\end{itemize}
\end{frame}
 
\end{document}

