\documentclass{beamer}
\usepackage[utf8]{inputenc}
\usepackage{amsmath, amssymb, amsthm, mathtools, gensymb}

\setbeamertemplate{footline}[frame number]
\beamertemplatenavigationsymbolsempty
\usecolortheme{seagull}

\newcommand{\tran}{\mathsf{T}}
\newcommand{\mc}[1]{\ensuremath{\mathcal{#1}}}
\newcommand{\mbb}[1]{\ensuremath{\mathbb{#1}}}

\title[Chapter 7]{Notes on Nocedal and Wright's ``Numerical Optimization''\\Chapter 8 --  ``Quasi-Newton Methods''}
\author{Lucas N. Ribeiro}
\date{}
 
\begin{document}
 
\frame{\titlepage}
 
\begin{frame}[allowframebreaks]{Introduction}
	\begin{itemize}
		\item First ideas: WC Davidon at Argonne National Lab in the 1950s
		\item Requires only gradient knowledge to achieve super-linear convergence
		\item Sometimes more efficient than Newton's method because it does not require 2nd order derivatives
	\end{itemize}
\end{frame}

\begin{frame}[allowframebreaks]{BFGS}

	\begin{block}{Highlights}
		\begin{itemize}
			\item BFGS iterates as $x_{k+1} = x_k + \alpha_k p_k$, where $p_k$ is obtained by minimizing a quadratic model at $x_k$, $p_k = -B_k^{-1}p_k$.
			\item Instead of recalculating a fresh $B_k$ at each step, it updates in a simple manner.
			\item The update is the solution of the \emph{secant equation}, which has solutions when the \emph{curvature condition} is satisfied
			\item The BFGS updates $B_k$ with a rank-$2$ matrix at each iteration
			\item Super-linear convergence
		\end{itemize}
	\end{block}

	Consider the quadratic model and its gradient
	\[
	m_k(p)= f_k + \nabla f_k^\tran p + 0.5 p^\tran B_k p,\quad \nabla m_k(p) = \nabla f_k + B_k p.
	\]

	\newpage
	\begin{block}{The secant equation}
		\begin{itemize}
			\item From the update formula, we define $s_k = x_{k+1} - x_k = B_{k+1} (\alpha_k p_k)$.
			\item A reasonable condition for BFGS is that the gradient of $m_{k+1}$ should match the gradient of $f$ at the latest two iterates $x_k$ and $x_{k+1}$. We have:
			\[
				\nabla m_{k+1}(-\alpha_k p_k) = \nabla f_{k+1} - \alpha_k B_{k+1} p_k \stackrel{!}{=} \nabla f_k
			\]
			\item Therefore:
			\[
				B_{k+1}(\alpha_k p_k) = \nabla f_{k+1} - \nabla f_k
			\]
			Defining $y_k=\nabla f_{k+1} - \nabla f_k$, we have the secant equation:
			\[
				B_{k+1} s_k = y_k.
			\]
		\end{itemize}
	\end{block}

	\textbf{Solving the secant equations}
	\begin{itemize}
		\item We wish to solve the secant equation $B_{k+1} s_k = y_k$ to nicely update our line search.
		\item Solving this system will be possible only if the curvature condition $s_k^\tran y_k >0$ (because then $B_{k+1}$ will be positive definite)
		\item When $f$ is strongly convex, then it is always satisfied.
		\item Otherwise, one has to be careful to enforce this condition on line search
	\end{itemize}

	When the curvature condition is satisfied, the system has in fact infinite solution. To find a single one, we impose additional conditions.

	\newpage
	
	We consider the following problem:
	\begin{subequations}
		\begin{gather}
			\min_B \| B- B_k \|\\
			\text{subject to}\quad B=B^\tran,\quad B s_k = y_k
		\end{gather}
	\end{subequations}

	\begin{itemize}
		\item The norm in this problem may be whatever. The weighted Frobenius norm gives an easy solution considering the average Hessian weight matrix.
		\item In this case, the unique solution to this problem gives the DFP formula:
		\begin{equation}
		B_{k+1} = (I- \gamma_k y_k s_k^\tran) B_k (I - \gamma_k s_k y_k^\tran) + \gamma_k y_k y_k^\tran
		\end{equation}
		where $\gamma_k = 1/(y_k^\tran s_k)$.
		\item Note that in order to calculate the step $p_k = -B_k^{-1} p_k$, we need the inverse of $B_k$.
		\item Define $H_k=B_k^{-1}$. Applying the Sherman-Morrison-Woodbury formula to $H_k$ gives: \[H_{k+1} = H_k - \frac{H_k y_k y_k^\tran H_k}{y_k^\tran H_k y_k} + \frac{s_k s_k^\tran}{y_k^\tran s_k}\]
		\item It's a rank-$2$ update!
		\item The DFP formula is effective, but was superseded by the BFGS formula
	\end{itemize}

	\newpage
	
	\begin{itemize}
		\item The BFGS formula is obtained by reformulating the secant equation as
		\begin{gather}
		H_{k+1} y_k = s_k
		\end{gather}
		\item (Note that we just left-multiplied the old version by $H_{k+1}$)
		\item To obtain a unique solution, we solve
		\begin{subequations}
			\begin{gather}
			\min_H \| H- H_k \|\\
			\text{subject to}\quad H=H^\tran,\quad H y_k = s_k
			\end{gather}
		\end{subequations}
		\item and the solution is:
		\begin{equation}
		H_{k+1} = (I-\rho_k s_k y_k^\tran) H_k (I - \rho_k y_k s_k^\tran) + \rho_k s_k s_k^\tran
		\end{equation}
		where $\rho_k = 1/(y_k^\tran s_k)$.
	\end{itemize}

	\newpage
	
	\begin{block}{BFGS}
		\begin{enumerate}
			\item Given: starting point $x_0$, conv. threshold $\epsilon$ and inverse Hessian approx. $H_0$ (identity matrix, for example)
			\item $k \leftarrow 0$
			\item While $\| \nabla f_k \| > \epsilon$
			\begin{itemize}
				\item Compute search direction $p_k = -H_k \nabla f_k$
				\item Update step $x_{k+1} = x_k + \alpha p_k$
				\item Compute $H_{k+1}$ by means of the BFGS formula
			\end{itemize}
		\end{enumerate}
	\end{block}

\end{frame}

\begin{frame}[allowframebreaks]{The SR1 method}
\end{frame}

\begin{frame}[allowframebreaks]{The Broyden class}
\end{frame}

\begin{frame}[allowframebreaks]{Convergence results}
\end{frame}

\end{document}

